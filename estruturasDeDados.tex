%Preambulo. 
\documentclass[12pt, onecolumn]{article}

%Pacotes. 
\usepackage[utf8]{inputenc}
\usepackage[brazil]{babel} 
\usepackage[top=2cm, bottom=2cm, right=2cm, left=2cm]{geometry}


%Inicio do documento.
\begin{document}

	\title{Estrutura de dados}
	\author{Thiago Figueiredo Marcos}
	\date{\today}
	%Inclusão do titulo.
	\maketitle

	\begin{abstract}
		Para permitir a organização dos dados na memória de maneira mais sofisticada
		é necessário estrutura de dados complexas que represente conjuntos de determinado
		dado e permita operações de manipulações de inserção e remoção por exemplo \\
		\textbf{Palavras chave: } Estrutura de dados, Linguagem C.
	\end{abstract}
	
	%Sumario. 
	\begin{center}
		\tableofcontents
	\end{center}

	\begin{center}
		\section{Introdução}
	\end{center}
		A compreensão das estrutura de dados são fundamentais para construção de soluções
		de problemas complexos, além da organização em si dos dados. No mundo analógico
		essas construções se dão de forma "natural", como as filas de pessoas, ou as pilhas
		de taréfas a serem feitas, também conseguimos ver no mundo abstrato a construção 
		dessas estruturas como por exemplo um conjunto de números reais representado em 
		uma estrutura de lista. Corriqueiramente representações do mundo real podem servir
		como analogia para organização e estruturação dos dados, um exemplo são as árvores.

	\section{Pilha}
		A política que orienta as manipulações desse tipo de estrutura é chamado de
		\textbf{LIFO}{\it Last in, First out}, ou seja, o último que entra é o primeiro
		que sai, uma analogia do mundo prático é imaginar uma pilha de pratos, em seguida, 
		posicionar seus olhos logo acima dos pratos, assim, só poderar retirar o que ve, que
		no caso, é o ultimo prato que foi inserido na pilha.
		\subsection{Inserção}
			A inserção do dado é feito obrigatoriamente no topo da pilha.
	
			Algoritmo padrão de inserção. 

			\begin{itemize}
				\item {\bf Cria-se} a estrutura que representa o seu dado.
				\item {\bf Aponte} o proximo dado para o topo.
				\item {\bf Topo } apontara para o proximo dado.
			\end{itemize}

		\subsection{Remoção}
			A remoção é feita obrigatoriamente no topo da pilha.

			Algoritmo padrão de remoção

			\begin{itemize}
				\item {\bf Salve} o topo atual.
				\item {\bf Aponte} o topo atual para o topo de baixo.
				\item {\bf Libere} a memoria do dado que estava no topo.
			\end{itemize}
\end{document}
